\let\negmedspace\undefined
\let\negthickspace\undefined
\documentclass[journal]{IEEEtran}
\usepackage[a5paper, margin=10mm, onecolumn]{geometry}
\usepackage{tfrupee} 

\setlength{\headheight}{1cm}
\setlength{\headsep}{0mm}     

\usepackage{gvv-book}
\usepackage{gvv}
\usepackage{cite}
\usepackage{amsmath,amssymb,amsfonts,amsthm}
\usepackage{algorithmic}
\usepackage{graphicx}
\usepackage{textcomp}
\usepackage{xcolor}
\usepackage{txfonts}
\usepackage{listings}
\usepackage{enumitem}
\usepackage{mathtools}
\usepackage{gensymb}
\usepackage{comment}
\usepackage[breaklinks=true]{hyperref}
\usepackage{tkz-euclide} 
\usepackage{listings}
\def\inputGnumericTable{}                                 
\usepackage[latin1]{inputenc}                                
\usepackage{color}                                            
\usepackage{array}                                            
\usepackage{longtable}                                       
\usepackage{calc}                                             
\usepackage{multirow}                                         
\usepackage{hhline}                                           
\usepackage{ifthen}                                           
\usepackage{lscape}
\usepackage{circuitikz}

\tikzstyle{block} = [rectangle, draw, fill=blue!20, 
    text width=4em, text centered, rounded corners, minimum height=3em]
\tikzstyle{sum} = [draw, fill=blue!10, circle, minimum size=1cm, node distance=1.5cm]
\tikzstyle{input} = [coordinate]
\tikzstyle{output} = [coordinate]

\begin{document}

\bibliographystyle{IEEEtran}
\vspace{3cm}

\title{Software Assignment Report}
\author{EE25BTECH11044 - Sai Hasini Pappula}
\maketitle

\renewcommand{\thefigure}{\thesection.\arabic{figure}}
\renewcommand{\thetable}{\thesection.\arabic{table}}
\numberwithin{equation}{section}
\numberwithin{figure}{section}
\numberwithin{table}{section}

\setlength{\intextsep}{10pt} 

\section{Summary of Singular Value Decomposition - Gilbert Strang}
\subsection{The SVD Formula}

Any matrix $A_{m \times n}$ can be decomposed as:
\[
A = U\Sigma V^{\top}
\]
where $U$ ($m \times m$) and $V$ ($n \times n$) are orthogonal matrices, and $\Sigma$ ($m \times n$) is diagonal with non-negative entries $\sigma_1 \geq \sigma_2 \geq \cdots \geq 0$ (singular values).

\subsection{Computing the SVD}

\subsubsection{Finding $V$ and $\Sigma$}
\begin{enumerate}
    \item Compute $A^{\top}A$ (an $n \times n$ matrix)
    \item Find eigenvectors of $A^{\top}A$ $\rightarrow$ columns of $V$
    \item Find eigenvalues $\lambda_i$ of $A^{\top}A$ 
    \item Singular values: $\sigma_i = \sqrt{\lambda_i}$
\end{enumerate}

\subsubsection{Finding $U$}
\begin{enumerate}
    \item Compute $AA^{\top}$ (an $m \times m$ matrix)
    \item Find eigenvectors of $AA^{\top}$ $\rightarrow$ columns of $U$
    \item Note: Eigenvalues of $AA^{\top}$ equal eigenvalues of $A^{\top}A$
\end{enumerate}

\subsection{Geometric Meaning}

The SVD establishes:
\[
Av_i = \sigma_i u_i
\]
where $v_i$ are orthonormal basis vectors in the row space, $u_i$ are orthonormal basis vectors in the column space, and $\sigma_i$ are stretching factors.

\subsection{Key Properties}

\begin{itemize}
    \item Orthogonality: $U^{\top}U = I$ and $V^{\top}V = I$
    \item Relationship: $Av_i = \sigma_i u_i$
    \item SVD works for \emph{any} matrix (rectangular, singular, etc.)
\end{itemize}

\section{Algorithm Explanation}

\subsection{Overview}

The implemented method performs a \textbf{Truncated Singular Value Decomposition (SVD)} 
using the \textbf{Power Iteration with Deflation} technique. 
This approach efficiently extracts the top-$k$ singular values and corresponding singular vectors of a real or complex matrix 
without computing a full SVD decomposition.

\subsection{Step-by-Step Procedure}

\begin{enumerate}
    \item \textbf{Form the covariance (or Hermitian) matrix:}
    \[
    B = A^{T} A \quad \text{(real case)} \qquad \text{or} \qquad B = A^{H} A \quad \text{(complex case)}
    \]
    The eigenvectors of $B$ correspond to the right singular vectors of $A$, and the eigenvalues of $B$ are the squared singular values of $A$.

    \item \textbf{Power iteration to find dominant eigenvector:} \\
    Start from a random vector $v$, and iteratively compute
    \[
    v_{t+1} = \frac{Bv_t}{\|Bv_t\|}
    \]
    until convergence. The resulting vector $v$ approximates the dominant eigenvector of $B$.

    \item \textbf{Compute the corresponding eigenvalue:}
    \[
    \lambda = v^{T} B v
    \quad \text{and} \quad
    \sigma = \sqrt{\lambda}
    \]
    where $\sigma$ is the singular value of $A$ associated with $v$.

    \item \textbf{Compute the left singular vector:}
    \[
    u = \frac{A v}{\|A v\|}
    \]
    The pair $(u, v)$ represents one singular vector pair.

    \item \textbf{Deflation:} \\
    To find the next singular component, remove the contribution of the current one:
    \[
    B = B - \sigma^{2} v v^{T}
    \]
    (or $B = B - \sigma^{2} v v^{H}$ for the complex case).
    This ensures the next power iteration targets the next dominant component.

    \item \textbf{Repeat for $k$ components:}
    Repeat steps 2 to 5 until $k$ dominant singular triplets $(u_i, \sigma_i, v_i)$ are found.

    \item \textbf{Reconstruct the rank-$k$ approximation:}
    \[
    A_k = \sum_{i=1}^{k} \sigma_i u_i v_i^{T}
    \]
    (or $A_k = \sum_{i=1}^{k} \sigma_i u_i v_i^{H}$ in the complex case).
\end{enumerate}

\subsection{Mathematical Intuition}

The Power Iteration method exploits a fundamental property of eigen decomposition:  
when a vector is repeatedly multiplied by a matrix, the component in the direction 
of the dominant eigenvector grows the fastest.

\subsubsection*{Power Iteration Convergence}

Let $B = A^{T}A$ (or $A^{H}A$ in the complex case), and let its eigen decomposition be
\[
B = V \Lambda V^{T}
\]
where $\Lambda = \text{diag}(\lambda_1, \lambda_2, \ldots, \lambda_n)$ with 
$\lambda_1 > \lambda_2 > \cdots > \lambda_n$.  
Starting from a random vector $v_0$, we repeatedly compute:
\[
v_{t+1} = \frac{B v_t}{\|B v_t\|}
\]
Expanding this recursively gives:
\[
v_t = \frac{B^t v_0}{\|B^t v_0\|} 
= \frac{V \Lambda^t V^{T} v_0}{\|V \Lambda^t V^{T} v_0\|}
\]
As $t \to \infty$, all terms involving smaller eigenvalues $\lambda_2, \lambda_3, \ldots$ 
decay relative to $\lambda_1^t$.  
Hence, $v_t$ converges to the eigenvector corresponding to the largest eigenvalue $\lambda_1$.

\subsubsection*{Deflation Principle}

Once the top eigenvector $v_1$ and corresponding singular value $\sigma_1 = \sqrt{\lambda_1}$ 
are found, their contribution is subtracted (deflated) from the covariance matrix:
\[
B = B - \sigma_1^2 v_1 v_1^{T}
\]
This removes the influence of the already extracted component, 
allowing the next iteration to converge to the eigenvector associated with the second-largest eigenvalue, and so on.

\subsubsection*{Relation to SVD}

For the original matrix $A$, each eigenpair $(\lambda_i, v_i)$ of $B = A^{T}A$ gives a 
right singular vector $v_i$ and corresponding left singular vector
\[
u_i = \frac{A v_i}{\|A v_i\|}
\]
with singular value $\sigma_i = \sqrt{\lambda_i}$.  
Thus, the Power Iteration with Deflation sequentially reconstructs the most significant 
singular triplets $(u_i, \sigma_i, v_i)$, forming a truncated approximation
\[
A_k = \sum_{i=1}^{k} \sigma_i u_i v_i^{T}
\]
that captures most of the energy of $A$ while reducing storage and computation.

\subsection{Error Computation}

The quality of the reconstruction is evaluated using the Frobenius norm error:
\[
\|A - A_k\|_F = \sqrt{\sum_{i=1}^{m}\sum_{j=1}^{n} (A_{ij} - (A_k)_{ij})^2}
\]
A smaller error indicates a more accurate reconstruction.

\subsection{Remarks}

This algorithm provides an efficient approximation of the top-$k$ singular components:
\begin{itemize}
    \item Avoids full matrix decomposition ($\mathcal{O}(n^3)$)
    \item Numerically stable and scalable
    \item Extendable to complex matrices using Hermitian operations ($A^H$)
    \item Ideal for image compression where few singular values capture most information
\end{itemize}

\subsection{Pseudocode for Truncated SVD using Power Iteration}

\noindent\textbf{Input:} Matrix $A \in \mathbb{R}^{m \times n}$ (or $\mathbb{C}^{m \times n}$), rank $k$ \\
\textbf{Output:} $U \in \mathbb{R}^{m \times k}$, $\Sigma \in \mathbb{R}^{k}$, $V \in \mathbb{R}^{n \times k}$

\begin{flushleft}
$B = A^{T} A$ \hfill (or $A^{H} A$ in the complex case) \\[3pt]
\textbf{for} $i = 1$ \textbf{to} $k$ \textbf{do} \\[2pt]
\hspace*{1.5em} Initialize random vector $v$ \\[2pt]
\hspace*{1.5em} \textbf{repeat until convergence:} \\[2pt]
\hspace*{3em} $w = Bv$ \\[2pt]
\hspace*{3em} $v = \dfrac{w}{\|w\|}$ \\[2pt]
\hspace*{1.5em} $\lambda = v^{T} B v$ \\[2pt]
\hspace*{1.5em} $\sigma = \sqrt{\lambda}$ \\[2pt]
\hspace*{1.5em} $\Sigma[i] = \sigma$ \\[2pt]
\hspace*{1.5em} $u = A v$ \\[2pt]
\hspace*{1.5em} $U[:, i] = \dfrac{u}{\|u\|}$ \\[2pt]
\hspace*{1.5em} $V[:, i] = v$ \\[2pt]
\hspace*{1.5em} $B = B - \sigma^{2} v v^{T}$ \hfill (or $B - \sigma^{2} v v^{H}$ for complex) \\[3pt]
\textbf{end for} \\[5pt]
\textbf{return} $U, \Sigma, V$
\end{flushleft}

\section{Comparison of Algorithms}

\begin{table}[H]
\centering
\begin{tabular}{|l|l|l|l|l|}
\hline
\textbf{Algorithm} & \textbf{Description} & \textbf{Complexity} & \textbf{Accuracy} & \textbf{Suitability} \\ \hline
Full Eigen Decomposition & Computes all eigenpairs of $A^TA$, $AA^T$ & $\mathcal{O}(n^3)$ & Exact & Slow for large images \\ \hline
Jacobi Eigenvalue Method & Iteratively zeroes off-diagonal elements & $\mathcal{O}(n^3)$ & High & Educational, slow \\ \hline
\textbf{Power Iteration (Used)} & Iteratively extracts top eigenvectors via deflation & $\mathcal{O}(k n^2)$ & Approximate & \textbf{Fast and practical} \\ \hline
\end{tabular}
\caption{Comparison of SVD computation methods.}
\end{table}

\subsection{Reason for Choosing the Algorithm}

Several algorithms can be used to compute the Singular Value Decomposition, such as 
Full Eigen Decomposition, the Jacobi Eigenvalue Method, and iterative methods like Power Iteration. 
Each method involves a trade-off between computational efficiency, accuracy, and implementation complexity.

\begin{itemize}
    \item \textbf{Full Eigen Decomposition} computes all eigenvalues and eigenvectors of 
    $A^{T}A$ (or $A^{H}A$ in the complex case) using dense matrix operations. 
    Although it provides an exact decomposition, its time complexity is $\mathcal{O}(n^3)$, 
    making it impractical for large images or matrices.

    \item \textbf{Jacobi Eigenvalue Method} iteratively zeroes out off-diagonal elements 
    using orthogonal transformations. 
    It achieves high accuracy but converges slowly and is computationally intensive for high-dimensional data. 
    It is more suitable for educational demonstrations or very small matrices.

    \item \textbf{Power Iteration with Deflation (Chosen)} is an iterative approach that 
    efficiently extracts only the top-$k$ dominant singular values and corresponding vectors. 
    It repeatedly multiplies the covariance matrix $B = A^{T}A$ with a random vector 
    until convergence, then removes the found component via deflation. 
    Its computational cost is approximately $\mathcal{O}(k n^2)$, 
    which is much lower than the $\mathcal{O}(n^3)$ cost of full decomposition.
\end{itemize}

\noindent
The Power Iteration with Deflation method was chosen because it provides a 
balance between computational efficiency and accuracy. 
It is especially effective for image compression tasks, where only a few 
dominant singular values capture most of the visual information. 
Moreover, its iterative nature makes it memory-efficient and suitable 
for both real and complex matrices.

\section{Reconstructed Images}

\begin{figure}[H]
\centering
\includegraphics[width=0.32\columnwidth]{figs/greyscale.png}
\includegraphics[width=0.32\columnwidth]{figs/globe.jpg}
\includegraphics[width=0.32\columnwidth]{figs/einstein.jpg}
\caption{Original images used for SVD compression.}
\end{figure}

\begin{figure}[H]
\centering
\includegraphics[width=0.32\columnwidth]{figs/output_k5.png}
\includegraphics[width=0.32\columnwidth]{figs/out_k5.png}
\includegraphics[width=0.32\columnwidth]{figs/ein_k5.png}
\caption{Reconstructed images for $k = 5$.}
\end{figure}

\begin{figure}[H]
\centering
\includegraphics[width=0.32\columnwidth]{figs/output_k20.png}
\includegraphics[width=0.32\columnwidth]{figs/out_k20.png}
\includegraphics[width=0.32\columnwidth]{figs/ein_k20.png}
\caption{Reconstructed images for $k = 20$.}
\end{figure}

\begin{figure}[H]
\centering
\includegraphics[width=0.32\columnwidth]{figs/output_k50.png}
\includegraphics[width=0.32\columnwidth]{figs/out_k50.png}
\includegraphics[width=0.32\columnwidth]{figs/ein_k50.png}
\caption{Reconstructed images for $k = 50$.}
\end{figure}


\section{Error Analysis}

\begin{table}[H]
\centering
\begin{tabular}{|c|c|l|}
\hline
\textbf{Rank (k)} & \textbf{Frobenius Error} & \textbf{Comment} \\ \hline
5   & 4713.590 & Large error, coarse approximation \\ \hline
20  & 2126.570 & Balanced quality vs compression \\ \hline
50  & 880.529  & Near-original reconstruction \\ \hline
100 & 164.920  & Very high quality, visually identical to original \\ \hline
\end{tabular}
\caption{Frobenius norm error for different truncation ranks $k$. 
As $k$ increases, the reconstruction quality improves significantly, approaching the original image for $k = 100$.}
\end{table}

\section{Discussion and Reflections}

\begin{itemize}
    \item Increasing $k$ improves image quality but reduces compression efficiency.
    \item The algorithm converged within 30--50 iterations per singular value.
    \item Regular normalization ensured numerical stability and convergence.
    \item Implementing the full method in C improved low-level understanding of SVD operations.
\end{itemize}

\end{document}
